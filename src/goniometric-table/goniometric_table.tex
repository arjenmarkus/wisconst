\documentclass[onecolumn]{article}
\usepackage{hyperref}
\begin{document}

\title{A mostly useless table for exact goniometric functions}

\author{Arjen Markus}

\maketitle

\subsection*{Introduction}
In school I learned the exact expressions for the sine, cosine and tangent
of certain angles -- 0, 30, 45, 60 and 90 degrees. These exact expressions
are fairly simple, but the list is incomplete: with the formulae for the
sine, cosine and tangent of the sum and the difference of two angles, it is easy
to get expressions for 15 and 75 degrees as well.

Here is the result:

\vspace{\baselineskip}
\begin{tabular}{ccccc}
\emph{Angle}     & \emph{Angle}          & \emph{Sine}                          & \emph{Cosine}                        & \emph{Tangent}              \\
\emph{degrees}   & \emph{radians}        &                                      &                                      &                             \\
\hline
   0             &     0                 &     0                                &     1                                &     0                       \\
  15             &  $\frac{1}{12} \pi$   &   $\frac{1}{4}(\sqrt{6} - \sqrt{2})$ &  $\frac{1}{4}(\sqrt{6} + \sqrt{2})$  &  $1 - \frac{1}{2} \sqrt{3}$ \\
  30             &  $\frac{1}{6} \pi$    &   $\frac{1}{2}$                      & $\frac{1}{2} \sqrt{3}$               &  $\frac{1}{3} \sqrt{3}$     \\
  45             &  $\frac{1}{4} \pi$    &   $\frac{1}{2} \sqrt{2}$             & $\frac{1}{2} \sqrt{2}$               &     1                       \\
  60             &  $\frac{1}{3} \pi$    &   $\frac{1}{2} \sqrt{3}$             & $\frac{1}{2}$                        &  $\sqrt{3}$                 \\
  75             &  $\frac{5}{12} \pi$   &   $\frac{1}{4}(\sqrt{6} + \sqrt{2})$ &  $\frac{1}{4}(\sqrt{6} - \sqrt{2})$  &  $2 + \sqrt{3}$             \\
  90             &  $\frac{1}{2} \pi$    &     1                                &     0                                &  $\infty$                   \\
\hline
\end{tabular}
\vspace{\baselineskip}

Can we refine this table? Yes, of course: we have the pentagon.

\subsection*{Extending the table via a pentagon}
The regular pentagon has angles of 72 degrees or $\frac{2}{5} \pi$ radians between the centre and its vertices (\url{https://mathworld.wolfram.com/RegularPentagon.html}). Thanks to this
figure, we know the exact sines and cosines of this angle:
\begin{eqnarray*}
    \cos(\frac{2}{5} \pi) &=& \frac{1}{4}(\sqrt{5} - 1) \\
    \sin(\frac{2}{5} \pi) &=& \frac{1}{4}\sqrt{ 10 + 2 \sqrt{5}}
\end{eqnarray*}

Together with the summation and difference formulae we can -- in principle -- extend the above table to multiples of 3~degrees!
In principle, because the expressions are rather complicated for some of these angles and lose all practical relevance.

The angle of 3~degrees $= 75 -72$~degrees has the following exact expressions for the sine and cosine:
\begin{eqnarray*}
     \sin(3^\circ)  &=& \sin(75^\circ) \cos(72^\circ) - \cos(75^\circ) \sin(72^\circ) \\
                    &=& \frac{1}{4} (\sqrt{6} + \sqrt{2}) \cdot \frac{1}{4}(\sqrt{5} - 1) - \frac{1}{4} (\sqrt{6} - \sqrt{2}) \cdot \frac{1}{4}\sqrt{ 10 + 2 \sqrt{5}} \\
     \cos(3^\circ)  &=& \cos(75^\circ) \cos(72^\circ) + \sin(75^\circ) \sin(72^\circ) \\
                    &=& \frac{1}{4} (\sqrt{6} - \sqrt{2}) \cdot \frac{1}{4}(\sqrt{5} - 1) + \frac{1}{4} (\sqrt{6} + \sqrt{2}) \cdot \frac{1}{4}\sqrt{ 10 + 2 \sqrt{5}} \\
\end{eqnarray*}

The possibilities to simplify these expressions are at best limited and the exact expression for $tan(3^\circ)$ is even more daunting!

For some of the multiples of 3~degrees (excluding the multiples of 15~degrees) there are relatively simple expressions, such as 12~degrees = $72 - 60$~degrees:
\begin{eqnarray*}
     \sin(12^\circ) &=& \sin(72^\circ) \cos(60^\circ) - \cos(72^\circ) \sin(60^\circ) \\
                    &=& \frac{1}{4}\sqrt{10 + 2 \sqrt{5}} \cdot \frac{1}{2} - \frac{1}{4}(\sqrt{5} - 1) \cdot \frac{1}{2} \sqrt{3} \\
                    &=& \frac{1}{8} \bigl(\sqrt{10 + 2 \sqrt{5}} - \sqrt{15} + \sqrt{3} \bigr)
\end{eqnarray*}
\noindent but even then it is completely impractical.

\end{document}
