\documentclass[onecolumn]{article}
\usepackage{graphicx}
\usepackage{float}
\usepackage{hyperref}
\restylefloat{figure}
\begin{document}

\title{Permutations, derangements and variants}

\author{Arjen Markus}

\maketitle

\section*{Some recurrences}
The topic of this note is something I stumbled upon while working on some simple combinatorical topics. As you know,
the number of permutations grows fast with the number of items. If you put some constraints on the permutations,
like requiring that no item remains in the same position, you still end up with a large number, but it is smaller
than a regular factorial number ($n!$). Permutations where no item occupies its original position are called
derangements (cf.\ \url{https://mathworld.wolfram.com/Derangement.html}). The number of derangements of n items, $d_n$, can be calculated via
a straightforward recurrence relation:
\begin{eqnarray}
\nonumber d_1     &=& 0 \\
\nonumber d_{n+1} &=& n d_n + (-1)^{n+1}
\end{eqnarray}

While this number does not grow quite as fast as a factorial, it tends to a fraction of the factorial:
\begin{eqnarray}
\nonumber lim_{n \rightarrow \infty} \frac{d_{n+1}}{n!} = e
\end{eqnarray}

But what happens if you change the recurrence relation a bit? Instead of an alternating 1 and -1, always add 1 or subtract 1?

It is easy enough to write a small program that calculates these numbers:
\begin{eqnarray}
\nonumber (n+1)!  &=& (n+1) n!           \\
\nonumber a_{n+1} &=& n a_n - 1          \\
\nonumber b_{n+1} &=& n b_n + 1          \\
\nonumber c_{n+1} &=& n c_n + max(0, (-1)^{n+1}) \\
\nonumber d_{n+1} &=& n d_n + (-1)^{n+1} \\
\end{eqnarray}

\noindent with initial conditions somewhat arbitrarily chosen as $a_1 = 1, b_1 = 1, c_1 = 0, d_1 = 0$.

The result for the first few values of n:

\begin{table}[h!]
\begin{tabular}{rrrrrr}
$n$     &    $n!$     &    $a_n$    &    $b_n$    &    $c_n$ &    $d_n$    \\
\hline
 1      &      1      &      1      &      1      &      0   &      0      \\
 2      &      2      &      1      &      3      &      1   &      1      \\
 3      &      6      &      2      &     10      &      3   &      2      \\
 4      &     24      &      7      &     41      &     13   &      9      \\
 5      &    120      &     34      &    206      &     65   &     44      \\
 6      &    720      &    203      &   1237      &    391   &    265      \\
 7      &   5040      &   1420      &   8660      &   2737   &   1854      \\
 8      &  40320      &  11359      &  69281      &  21897   &  14833      \\
 9      & 362880      & 102230      & 623530      & 197073   & 133496      \\
\end{tabular}
\end{table}

Calculating the ratios with $n!$ gives the following results (rounded to four decimals):
\begin{eqnarray}
a_n / n! & \rightarrow & 0.2817  \\
b_n / n! & \rightarrow & 1.718   \\
c_n / n! & \rightarrow & 0.5431  \\
d_n / n! & \rightarrow & 0.3679
\end{eqnarray}

The convergence is very fast -- it takes only eight steps to reach the end result in four decimals.
We can see that $d_n$ (the number of derangements) converges to $1/e$, as expected. We can see that $a_n$ likely converges
to $3-e$ and $b_n$ to $e-1$, judging from the numerical values.

The last one, $c_n$, is a bit more difficult, but:
\begin{eqnarray}
\frac{0.5431}{e} & \approx & 0.19979
\end{eqnarray}
\noindent so one could expect the exact result to be $\frac{1}{5} e$.

\section*{Analysis}
The somewhat empirical considerations need to be substantiated and luckily it is not too difficult to
analyse the results in a more thorough way. Let us take the following general relation:
\begin{eqnarray}
    A_1 &=& ... \\
    A_{n+1} &=& (n+1) A_n + k_n
\end{eqnarray}

The ratio $A_{n+1} / (n+1)!$ can be written as:
\begin{eqnarray}
    \frac{A_{n+1}}{(n+1)!} &=& \frac{(n+1) A_n + k_n}{(n+1) n!} \\
                           &=& \frac{A_n}{n!} + \frac{k_n}{(n+1)!}
\end{eqnarray}

Continuing this process we get the equation:
\begin{eqnarray}
    \frac{A_{n+1}}{(n+1)!} &=& A_1/1! + \sum_{i=2}^n \frac{k_i}{(i+1)!}
\end{eqnarray}

For $k_n = 1$ ($b_n$ in the definition above), this leads to:
\begin{eqnarray}
    lim_{n \rightarrow \infty} \frac{b_{n+1}}{(n+1)!} &=& 1 + \sum_{i=2}^n \frac{1}{(i+1)!} \\
                                                     &=& 1 + e-2 ~=~ e-1
\end{eqnarray}

For $k_n = -1$ ($a_n$), we get:
\begin{eqnarray}
    lim_{n \rightarrow \infty} \frac{a_{n+1}}{(n+1)!} &=& 1 - \sum_{i=2}^n \frac{1}{(i+1)!} \\
                                                      &=& 1 - (e-2) ~=~ 3 - e
\end{eqnarray}

Now the interesting case is $c_n$. The series is:
\begin{eqnarray}
    lim_{n \rightarrow \infty} \frac{c_{n+1}}{(n+1)!} &=& 0 + \frac{1}{2!} + \frac{1}{4!} + \frac{1}{6!} + ...
\end{eqnarray}
\noindent so with the odd terms cancelled. If we add the series for $e$ and $e^{-1}$, then we get a cancellation
of these terms as well, so that (compensating for the first term, $1/0!$:
\begin{eqnarray}
    lim_{n \rightarrow \infty} \frac{c_{n+1}}{(n+1)!} &=& \frac{1}{2} \left(e + e^{-1} - 2 \right)
\end{eqnarray}

Numerically, the value is 0.5430806..., quite close to $\frac{1}{5} e = 0.543656...$. The expectation that the
ratio converges to $\frac{1}{5} e$ is plainly the consequence of numerical similarity only.


\end{document}

