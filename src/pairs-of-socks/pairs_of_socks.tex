\documentclass[onecolumn]{article}
\usepackage{graphicx}
\usepackage{float}
\usepackage{amsmath}
\restylefloat{figure}
\begin{document}

\title{Pairs of socks}

\author{Arjen Markus}

\maketitle

\section*{Introduction}
Consider the following simple combinatorial problem: you have a drawer full of
socks and you blindly draw a number of them. What is the probability of drawing
a complete pair? What happens to the probabibilities if you need three of a kind
or four of a kind?

We assume that the pairs of socks are all complete, similarly for the objects that
come in three or four or ... Of course, we might also consider the case where there
is a only a single sock of a pair, but that is beyond the current investigation.

\section*{Straightforward observations}
If you have $n$ sets of items, then drawing $n+1$ items guarantees that you have at
least one complete pair (or in general: one complete set). But the larger the number
of sets. the more unlikely it will become to get a complete set with a smaller
number.

Let us consider two pairs of socks for the moment, a pair of red socks and a pair
of blue socks. Then taking two at random from the drawer leads to three possibilities:
\begin{itemize}
\item
Two red socks
\item
Two blue socks
\item
One red and one blue sock. Since there are two such mixed pairs, this possibility
is twice as likely.
\end{itemize}

Hence, we can conclude that in this case drawing a complete pair has a probability
of $\frac{1}{2}$. Another line of reasoning, however, leads to $\frac{1}{3}$:
\begin{itemize}
\item
Draw the first sock, it is either red or blue, that does not matter.
\item
We are left with one sock of the right colour and two socks of the wrong colour.
\item
The probability of drawing a sock of the right colour, and thereby obtaining a complete
pair, is therefore $\frac{1}{3}$.
\end{itemize}

I have no explanation for this discrepancy, other than that combinatorial problems
can surprise us. In any case, with the general cases, it is easier and more reliable
to use combinatorial formulae and visualisation techniques.

\section*{The general case: n pairs of socks}
If we have $n$ pairs, then the total number of socks is $2n$, so that the number $N$ of
possible combinations becomes:
%
\begin{eqnarray}
\nonumber N &=& \binom{2n}{n} \\
\nonumber   &=& n(2n-1)
\end{eqnarray}

The number of combinations that have a complete pair is, of course, $n$, as there
are $n$ pairs. So, the probability of getting a complete pair is $1/(2n-1)$.

Now, what happens if we draw $m$ socks in stead of 2?

We again have the total number of combinations and the number of acceptable
combinations -- even if we draw two socks of the same pair rightaway, we can continue
drawing socks up to the total of $m$. The rest does not matter, but the counting
is much simpler.

The total number of combinations is:
%
\begin{eqnarray}
\nonumber N &=& \binom{2n}{m}
\end{eqnarray}
%
Na\"ive reasoning leads to us to conclude that the number of acceptable combinations $M$ is:
\begin{eqnarray}
\nonumber M &=& n \binom{2n-2}{m-2}
\end{eqnarray}
%
\noindent as the $m-2$ extra socks that are drawn from the remaining $2n-2$ do not matter.

To illustrate this, consider the case of three pairs of socks, numbered 1 to 6. Socks 1 and 2
form a pair, 3 and 4 are the second pair and 5 and 6 the third. Then the combinations
possible with three arbitrary socks are:

1, 2, 3 -- pair 1

1, 2, 4 -- pair 1

1, 2, 5 -- pair 1

1, 2, 6 -- pair 1

1, 3, 4 -- pair 2

1, 3, 5

1, 3, 6

1, 4, 5

1, 4, 6

1, 5, 6 -- pair 3

2, 3, 4 -- pair 2

2, 3, 5

2, 3, 6

2, 4, 5

2, 4, 6

2, 5, 6 -- pair 3

3, 4, 5 -- pair 2

3, 4, 6 -- pair 2

3, 5, 6 -- pair 3

4, 5, 6 -- pair 3

Of the total of 20 combinations 12 contain a pair.

In general, the probability $p$ of retrieving at least one pair is:
%
\begin{eqnarray}
\nonumber p &=& n \binom{2n-2}{m-2} \bigg/ \binom{2n}{m}
\end{eqnarray}

Filling in $n = 3, m = 3$, we get:

\begin{eqnarray}
\nonumber p &=& 3 \binom{4}{1} \bigg/ \binom{6}{3} = 3 \cdot 4 / 20 = \frac{3}{5}
\end{eqnarray}

Filling in $n = 2, m = 3$, we get:

\begin{eqnarray}
\nonumber p &=& 2 \binom{2}{1} \bigg/ \binom{4}{3} = 2 \cdot 2 / 4 = 1
\end{eqnarray}

But filling in $n =2, m = 4$, we get:

\begin{eqnarray}
\nonumber p &=& 2 \binom{2}{2} \bigg/ \binom{4}{4} = 2 \cdot 1 / 1 = 2 > 1
\end{eqnarray}

What we have overlooked is the fact that with more than three socks to draw,
sometimes we will have a combination with two or even more complete pairs.
Hence we counting several combinations twice or three times.

To solve this problem we need a different approach: let us look at the combinations
that certain do not contain a single complet pair and subtract that from the
total number of combinations.

Observe that drawing the first sock will give us an incomplete pair and there
are $2n$ possibilities to do so. Drawing the second sock should be done on the set where
the matching sock is not included. Thus, we have $2n-2$ possibilities for drawing the
next sock. For the third sock we need again to exclude the remaining matching sock,
so we get to draw from $2n-4$ socks, and so on. With this process the number of
possibilities relies on the permutation of all these single socks, so we need to
correct for the number of permutations of $m$ objects.

In general the number $M_*$ of ways to draw $m$ socks without forming a pair is:

\begin{eqnarray}
\nonumber M_* &=& 2n \cdot (2n-2) \cdot (2n-4) ... (2n-2m+2) / m! \\
            &=& 2^m \binom{n}{m}
\end{eqnarray}

(A quick check on the correctness of this formula: selecting one sock from $2n$ socks
gives $2n$ possibilities -- $2 \cdot \binom{n}{1} = 2n$)

Hence the formula for the probability of getting at least one pair by drawing
$m$ socks from $n$ pairs is:

\begin{eqnarray}
\nonumber p &=& 1 - 2^m \binom{n}{m} \bigg/ \binom{2n}{m}
\end{eqnarray}


\section*{How many socks to draw?}
With the formula we can estimate how many socks you would have to draw
to be reasonably sure that there is at least one complete pair among them.
Let us define this as a probability equal to or larger than $\frac{1}{2}$.

The table below gives the estimate found in this way.

\begin{table}[h!]
\caption{The number of socks to draw for a probability $\geq \frac{1}{2}$.}
\center
\begin{tabular}{cc|cc}
  $n$       &   $m$   &    $n$       &  $m$   \\
\hline
   2        &    3    &     9        &    5   \\
   3        &    3    &     10       &    6   \\
   4        &    4    &     15       &    7   \\
   5        &    4    &     20       &    8   \\
   6        &    4    &     25       &    9   \\
   7        &    5    &     30       &    9   \\
   8        &    5    &    100       &   17   \\
\end{tabular}
\end{table}


\section*{Extension to sets of three or more objects}
If, instead of pairs, we have objects that come in triples or quadruples, the
formulae will change, but the line of reasoning will not.

So, the number of combinations $N$ when selecting $m$ objects from $3n$ triples, is
($m$ must be at least 3, as otherwise we cannot have a full triplet):

\begin{eqnarray}
\nonumber N &=& \binom{3n}{m}
\end{eqnarray}

The number of combinations that are guaranteed not to contain a full triplet is:

\begin{eqnarray}
\nonumber M_* &=& 3n \cdot (3n-3) \cdot (3n-6) ... (3n-3m+3) / m! \\
            &=& 3^m \binom{n}{m}
\end{eqnarray}

And therefore the probability of drawing a full triplet from a set of $n$ triplets
at random is:

\begin{eqnarray}
\nonumber p &=& 1 - 3^m \binom{n}{m} \bigg/ \binom{3n}{m}
\end{eqnarray}

The general case of k-tuplets:

\begin{eqnarray}
\nonumber N &=& \binom{kn}{m}
\end{eqnarray}

\begin{eqnarray}
\nonumber M_* &=& kn \cdot (kn-k) \cdot (kn-2k) ... (3n-km+k) / m! \\
            &=& k^m \binom{n}{m}
\end{eqnarray}

And therefore the probability of drawing a full triplet from a set of $n$ triplets
at random is:

\begin{eqnarray}
\nonumber p &=& 1 - k^m \binom{n}{m} \bigg/ \binom{kn}{m}
\end{eqnarray}


\end{document}
